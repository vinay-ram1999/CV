%\subsection{WebScraping Using R \textnormal{ | \textit{R, rvest, ggplot} \hfill Fall 2023}}
%\textit{Capstone Project} | Data Analytics Using R Programming Course \hfill \faCodeFork\ \href{https://github.com/vinay-ram1999/WebScraping-R}{Github}
%\begin{zitemize}
%    \item 
%\end{zitemize}


\subsection{WebScraping Using R \textnormal{ | \textit{R, rvest, ggplot} \hfill Spring 2024}}
\textit{Capstone Project} | Data Analytics Using R Programming Course \hfill \faCodeFork\ \href{https://github.com/vinay-ram1999/WebScraping-R}{Github}
\begin{zitemize}
    \item Developed a web scraping tool using the rvest package in R to collect articles (2008-2024) from an open-access journal ``Parasites \& Vectors''
    \item Implemented a unique scraping approach that minimizes resource usage by only fetching newly added articles after the initial data scrape, significantly reducing computational costs and time
    \item Cleaned the raw data using techniques such as regular expressions (regex) and conducted exploratory data analysis (EDA) using the ggplot
\end{zitemize}


\subsection{eComputer Store Database System \textnormal{ | \textit{MySQL, Python, Streamlit, Pandas} \hfill Fall 2023}}
\textit{Capstone Project} | Data Managment Systems Design Course \hfill \faCodeFork\ \href{https://github.com/vinay-ram1999/CS631-DMSD}{Github}
\begin{zitemize}
    \item Developed a comprehensive database system for an e-commerce store using MySQL for back-end management, following a structured approach including the creation of ER diagrams and transforming them into a relational database schema
    \item Built an interactive web-based user interface using Python's Streamlit library, enabling users to query and interact with the database
    \item Deployed the system on Github Pages, providing seamless access to the database for users through a fully functional web application
\end{zitemize}


\subsection{AlgoTrade API \textnormal{ | \textit{Python, yfinance, Pandas, Tensorflow, ks-api-client} \hfill 2023}}
\textit{Personal Project} \hfill \faCodeFork\ \href{https://github.com/vinay-ram1999/AlgoTrade-API}{Github}
\begin{zitemize}
    \item Developed a fully automated trading bot for NSE stocks using Python, integrating real-time and historical data with the yFinance library
    \item Implemented technical analysis by calculating technical indicators such as Moving Average, MACD, Bollinger Bands etc., with various trading strategies based on these indicators to automate buy/sell decisions
    \item Trained machine learning models, including LSTM, using historical stock data to predict and forecast stock prices
    \item Integrated Kotak Securities API (ks-api-client) for executing live trades based on the generated strategies
\end{zitemize}


%\subsection{MURPH Analytics \textnormal{ | \textit{Python, Streamlit, Pandas, Matplotlib, Plotly} \hfill Mar 2023 - May 2023}}
%{\textit{Personal Project} \hfill \href{https://murph-postprocess.streamlit.app}{Github}}
%\begin{zitemize}
%\item {Built an interactive dashboard using Streamlit to post-process and perform different analytical studies on the simulation data generated from the MURPH solver}
%\item {Implemented unstructured mesh plotting and 2D contour plots using Plotly}
% \item {Implementation of 3D plotting and grid independence study }
%\end{zitemize}


%\subsection{MURPH \textnormal{ | \textit{Python, SymPy, NumPy, OpenMP, CUDA, OP2-API, CFD} \hfill Dec 2022 - Ongoing}}
%{\textit{In collaboration with Dr. Satya P. Jammy (Associate Professor, Mechanical Department, SRM University AP)} \hfill \href{https://github.com/UnstructuredFVM}{Github}}
%\begin{zitemize}
%\item {A Multi-dimensional (2-D/3-D), Unstructured-mesh and Reactions based Parallel solver for Hyper-sonic flow regimes (MURPH)}
%\item {New design approach is used to build the solver code in python, so that the parallel code generation can be fully automated}
%\item {A 2D Ideal gas CFD solver is developed, capable of running parallel on multicore CPUs and GPUs using OpenMP and CUDA respectively}
%\item {Work is underway for integrating species reactions and development of a 3D solver}
%\end{zitemize}

%====================
% EXPERIENCE A
%====================
%\subsection{Accelerating a Hypersonic $CO_2$ Reaction Solver \textnormal{ | \textit{Python, SymPy, NumPy, OpenMP, CUDA, OP2-API, CFD} \hfill Jul 2020 - May 2021}}
%{\textit{B.Tech Final Year Project} \hfill \href{https://vinay-ram1999.github.io/files/UG_thesis.pdf}{Github}}
%\begin{zitemize}
%\item {A 2-D unstructured finite volume method based solver capable of simulating hypersonic flows in $CO_2$ rich Martian atmosphere with chemical kinetics for eight species consisting of $N_2 , O_2 , NO, O, N, CO_2 , CO$ and $C$ incorporated in it is developed using pYthon}
%\item {The solver is parallelized to run on single or multiple CPUs and Nvidia GPUs in order to attain greater computational speeds using OP2 application programming interface}
%\end{zitemize}

%====================
% EXPERIENCE B
%====================
%\subsection{Design and Analysis of Heat Sink \textnormal{ \hfill Oct 2020 - Dec 2020}}
%{\textit{Thermal Design of Electronic Equipment Course} \hfill Coursework Project}
%\begin{zitemize}
%\item {Designed and analysed a 2-D heat sink made of Aluminium and a layer of epoxy between the chip and heat sink with specified thermal properties as a thermal design solution for cooling a noval chip (CPU for example) using CFD-HT (Computational Fluid Dynamics and Heat Transfer) software ANSYS Fluent}
%\end{zitemize}


%====================
% EXPERIENCE C
%====================
% \subsection{Flow through an Exhaust Manifold \textnormal{ \hfill Feb 2020 - May 2020}}
% {\textit{Computational Fluid Dynamics Course} \hfill Coursework Project}
% \begin{zitemize}
% \item {Designed an exhaust manifold for a V-8 engine in Solidworks and then imported this geometry into the Ansys workbench}
% \item {Using Fluent, simulated the flow of the hot exhaust gases coming out from the engine and designed various diffuser models and studied the flow behavior for different diffuser models and submitted the final report for the optimal design}
% \end{zitemize}


%====================
% EXPERIENCE D
%====================
% \subsection{Sub-Sonic Wind tunnel \textnormal{ \hfill Sept 2019 - Dec 2019}}
% {\textit{Undergraduate Research Program (UROP)} \hfill Academic Project}
% \begin{zitemize}
% \item {Developed a Subsonic wind tunnel, which is designed to achieve 30m/s in the test section with expected low intensity turbulence level, making it available for researching in areas such as low speed aerodynamics and other such possibilities}
% \end{zitemize}
